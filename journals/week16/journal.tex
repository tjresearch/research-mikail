% Syslab Research Journal Template
% By Patrick White
% September 2019
% version 1.1 - 9/5/2019

% INSTRUCTIONS: Edit the file as appropriate and replace with your journal text. Do NOT edit
%							any section headers or titles, tabling commands, fonts, spacing, sizes, etc.

% -------  Do NOT edit this header
\documentclass[letterpaper,11pt]{article}
\usepackage[paperwidth=8.5in,left=1.0in,right=1.0in,top=1.0in,bottom=1.0in,paperheight=11.0in]{geometry}
\usepackage{graphicx}
\graphicspath{ {./} }
\usepackage{palatino}
\def\hrulefill{\leavevmode\leaders\hrule height 20pt\hfill\kern\z@}

% ------------- DO Edit these definitions ---------------------
\def\name{Mikail Khan}
\def\period{5}
\def\journalnum{16}
\def\daterange{1/27/20-2/07/20}
% ------------------ END ---------------------------------

% Do NOT edit this
\begin{document}
	\thispagestyle{empty}
	\begin{flushright}
		{\Large Journal Report \journalnum} \\
		\daterange\\
		\name \\
		Computer Systems Research Lab \\
		Period \period, White
		\end{flushright}
	\hrule height 1pt
% ------------------ END ---------------------------------%	
	
% ------------------- Begin Journal reporting HERE ---------------

% ------ SECTION DAILY LOG -------------------------------------
	\section*{Daily Log}

	\vspace{-0.5em}
		\subsection*{Monday January 27}

                I fixed a bug, cleaned up some code, and added a red outline to selected bodies.

		\subsection*{Wednesday January 29}

                I added basic acceleration graphing in the same way as I did with the mechanics sim. The opacity of older velocities decreases so the brightest one is the current one, and it only saves the last 25 frames. Graphing the acceleration of an orbiting body looks like radar. I also refactored the gui library wrapper and moved more stuff from the sidepanel to the top gui bar.
                
		\subsection*{Friday January 31}

                Honestly the acceleration graph is super boring and it's practically the same as an FBD. Adding FBDs would also increase the complexity of the graphing systems a lot. Instead, I decided to just polish the UI a bunch. I made the top bar a lot more important, so that now the sidepanel is only used with selected entities. I think it's a lot simpler and easy to understand now. I also added a slight radius around the mouse so that you don't have to click exactly on a body to select it. The last thing I added was a help menu.

                \subsection*{Monday February 3}

                Started planning out wave simulator. There will be shapes other than circles but I don't really need access to the details of the collision, so I'm not sure if I should use a collision detection library since they're annoying to use. On the other hand, there will likely be lots of circles, one for each wavelet, so I might have to use one for performance's sake. Other than that, the general structure of the wave sim is pretty similar to the gravity sim.

                \subsection*{Wednesday February 5}

                Did some more planning and wrote most of the boilerplate. I did decide to use a collision detection library because I suppose that having to write a collision detection algorithm even between a circle and a triangle might take more work than using the library.

                \subsection*{Friday February 7}

                Presentations

	
% ------ SECTION TIMELINE -------------------------------------
	\newpage
	\section*{Timeline}
	\begin{tabular}{|p{1in}|p{2.5in}|p{2.5in}|}
		\hline 
	\textbf{Date} & \textbf{Goal} & \textbf{Met}\\ \hline
		\hline
		Today minus 2 weeks & Polish gravity GUI & Yes\\
		\hline
		Today minus 1 weeks & FBDs & Acceleration graphs but no FBDs \\
		\hline
		Today  & Start wave lab sim & Most of the ECS/boilerplate is done \\
		\hline
                Today plus 1 week & Get very basics of wavelets and collisions done & The collision detection library adds some startup work\\
		\hline
		Today plus 2 weeks & Working wave simulation & Waves should work but limited access to different emitters etc.  \\
		\hline
	\end{tabular}


% ------ SECTION REFLECTION  -------------------------------------
	\section*{Reflection}
			In narrative style, talk about your work this week. Successes, failures, changes to timeline, goals. This should also
			include concrete data, e.g. snippets of code, screenshots, output, analysis, graphs, etc.

                        I think I'm done with the gravity sim for now. I thought FBDs would be interesting but now I realize that they'd basically all be the same. I still want to add them to the mechanics sim though, although I'll have to fix simultaneous collisions there first. The GUI is fairly nice to use now, and I've added a few keyboard shortcuts. There are still some things that are a bit annoying, but they'd take a while to fix and they're just supplemental. I'm looking forwards to starting the wave lab sim. I think it will be fairly simple but I'm not sure how to group wavelets into waves, or if that's even necessary. I have written up a rough plan though. I've also decided to complexity rewrite the mechanics sim in my freetime because I want to.
                        I was wondering how I'd handle rendering the waves because even if a bunch of circles as wavelets can visually demonstrate what's happening they would look pretty un-wavelike. Recently I decided to learn vulkan to write a super simple game engine so I wrote a few shaders so I think I'll try to write a shader to turn the circles into a more or less continuous line. The game engine I'm using does support custom shaders and since I just need one that's applied universally I think it should be fairly easy to get that part working the hard part will be writing the shader.
\end{document}
