% Syslab Research Journal Template
% By Patrick White
% September 2019
% version 1.1 - 9/5/2019

% INSTRUCTIONS: Edit the file as appropriate and replace with your journal text. Do NOT edit
%							any section headers or titles, tabling commands, fonts, spacing, sizes, etc.

% -------  Do NOT edit this header
\documentclass[letterpaper,11pt]{article}
\usepackage[paperwidth=8.5in,left=1.0in,right=1.0in,top=1.0in,bottom=1.0in,paperheight=11.0in]{geometry}
\usepackage{palatino}
\def\hrulefill{\leavevmode\leaders\hrule height 20pt\hfill\kern\z@}

% ------------- DO Edit these definitions ---------------------
\def\name{Mikail Khan}
\def\period{5}
\def\journalnum{1}
\def\daterange{9/2/19-9/6/19}
% ------------------ END ---------------------------------

% Do NOT edit this
\begin{document}
	\thispagestyle{empty}
	\begin{flushright}
		{\Large Journal Report \journalnum} \\
		\daterange\\
		\name \\
		Computer Systems Research Lab \\
		Period \period, White
		\end{flushright}
	\hrule height 1pt
% ------------------ END ---------------------------------%	
	
% ------------------- Begin Journal reporting HERE ---------------

% ------ SECTION DAILY LOG -------------------------------------
	\section*{Daily Log}

	\vspace{-0.5em}
		\subsection*{Monday September 2}

                I tried to directly apply gravity to my rotational/translational simulator by adding a downward velocity (as opposed to a force). It didn't work very well and made energy increase a lot between collisions, making it pretty funny to watch. I read a wikipedia article about modeling collisions through impulses, which is what I'm doing, but it has a slightly different implementation which I think will make things a lot easier.

		\subsection*{Wednesday September 4}

                I tried implementing impulse based collisions as described by the wikipedia article, but I was getting some weird results and I didn't completely understand the math. I looked at two of the sources of the article though and they explained it a bit better so I got it to work. There were still some problems with applying gravity but having a distinction between using impulses as instantaneous forces and forces that are actually affected by the timestep helped. Something I noticed is that lower elasticities don't decrease the energy of the system, so I added a simple velocity *= elasticity after each collision but based purely on my own expectations it seems to have too strong of an effect.

		\subsection*{Friday September 6}

                I worked a bit more with how impulses are applied vs how forces are applied. I got gravity to mostly work except that objects resting on the ground jitter around a lot, so I decided to make a distinction between collisions and resting forces. It seems to work pretty well but how I'm deciding between an impulse collision and a resting force is kind of arbitrary. 

	
% ------ SECTION TIMELINE -------------------------------------
	\newpage
	\section*{Timeline}
	\begin{tabular}{|p{1in}|p{2.5in}|p{2.5in}|}
		\hline 
	\textbf{Date} & \textbf{Goal} & \textbf{Met}\\ \hline
		\hline
		Today minus 2 weeks & Get elastic translational collisions working & Yes, but it turns out that I can’t really reuse much code for the rotational part of it\\
		\hline
		Today minus 1 weeks & Rotational elastic collisions & Mostly, I got it working that weekend \\
		\hline
		Today  & Persistent collisions and gravity/friction & Persistent collisions and gravity, but not friction\\
		\hline
		Today plus 1 week & Functional 2D mechanics with basic input & Should be pretty fun \\
		\hline
                Today plus 2 weeks & More complete GUI, easy to use, but not with FBDs, graphs, etc. & It looks the milestone's I've set in these reports are way ahead of the ones I put in the proposal. I'd like to refactor/clean up a bit. \\
		\hline
	\end{tabular}


% ------ SECTION REFLECTION  -------------------------------------
	\section*{Reflection}
			In narrative style, talk about your work this week. Successes, failures, changes to timeline, goals. This should also
                        include concrete data, e.g. snippets of code, screenshots, output, analysis, graphs, etc.\\[5mm]

                        Even though I didn't quite meet my goal of adding friction, I think I was pretty successful. Honestly I had to rewrite a lot of the collisions stuff I did last week but it still helped a lot. I didn't realize how far ahead I was compared to the timeline I had made before, so I think I'll spend some time refactoring and generally cleaning up my code since it's pretty messy right now. I also have a lot of time to work on the UI and graphs. I would like to spend some time optimizing it, but I'm not sure what improvements I can realistically make. \\

                        The simulation itself has become presentable I think. It's hard to know if it's 100\% accurate, but everything looks fine and I think I'll try to make it into more modular framework outside of class so that I can use it for games. \\

                        I don't think screenshots would be very informative and I don't have any graphs yet, but here's my code that calculates the magnitude of the impulse. It's in Rust.

   \begin{verbatim}
      let jr_mag: f64 = {
            let res_vel = -(1.0 + elasticity) * vr;
            let numerator = na::Matrix::dot(&res_vel, &normal);

            let mut denominator = {
                let m_inv_sum = 1.0/m1 + 1.0/m2;

                // radius cross normal z component
                // r x n
                // dbg!(r1, r2);
                let ang_z_comp1 = (r1.x * normal.y - r1.y * normal.x) / I1;
                let ang_z_comp2 = (r2.x * normal.y - r2.y * normal.x) / I2;
                // dbg!(ang_z_comp1, ang_z_comp2);

                //the above crossed with r again
                // (r x n) x r
                let z_comp_moment_1 = Vector::new(-r1.y * ang_z_comp1, r1.x * ang_z_comp1);
                let z_comp_moment_2 = Vector::new(-r2.y * ang_z_comp2, r2.x * ang_z_comp2);
                // dbg!(z_comp_moment_1, z_comp_moment_2);

                let ang_sum = z_comp_moment_1 + z_comp_moment_2;

                let ang_dot_n = na::Matrix::dot(&ang_sum, &normal);
                ang_dot_n + m_inv_sum
            };

            numerator/denominator
        }.abs();

    \end{verbatim}

    This is how I initialize objects:

    \begin{verbatim}
     let obj_data = [
        ObjectData::Rectangle([10.0, 0.], [0., 0.], 19.5, 0.2).with_mass(INFINITY),
        ObjectData::Rectangle([10.0, 20.], [0., 0.], 19.5, 0.2).with_mass(INFINITY),
        ObjectData::Rectangle([0.0, 10.], [0., 0.], 0.2, 19.5).with_mass(INFINITY),
        ObjectData::Rectangle([20.0, 10.], [0., 0.], 0.2, 19.5).with_mass(INFINITY),
        // ObjectData::Circle([0.0, 0.0], [0., 0.], 1.),
        // ObjectData::Circle([18., 10.], [-2., 0.], 1.),
        // ObjectData::Square([0., 10.], [0.5, 0.], 1.5).rotate(PI/4.),
        // ObjectData::Square([18., 13.], [-0.5, 0.], 1.).rotate(PI/4.),
        // ObjectData::Circle([18., 18.], [-0.5, -0.2], 1.0),
        // ObjectData::Circle([0., 12.], [0.2, 0.1], 0.5),
        ObjectData::Circle([2.5, 15.], [0.12, 0.0], 1.0).with_mass(15.0),
        ObjectData::Rectangle([5., 12.0], [0.1, 0.2], 1.0, 1.0).with_mass(6.0).with_color([20, 240, 20]),
        ObjectData::Rectangle([15.0, 15.0], [-0.25, -0.0], 2.0, 2.0).with_mass(25.0),
        // ObjectData::Rectangle([5.0, 5.], [-0.5, 0.], 1.0, 2.25).rotate(PI / 5.).with_mass(25.0),
        // ObjectData::Rectangle([15.0, 11.], [-0.5, 0.0], 1.0, 1.25).with_mass(12.0),
        ObjectData::Rectangle([3.0, 5.], [0.0, -0.1], 1.0, 2.25).with_mass(15.0).rotate(0.2),
        // ObjectData::Circle([18., 11.0], [-0.85, 0.0], 0.5),
        // ObjectData::Convex([10.0, 10.], [-2., 0.], &points).rotate(PI / 3.),
    ];
    \end{verbatim}
\end{document}
