% Syslab Research Journal Template
% By Patrick White
% September 2019
% version 1.1 - 9/5/2019

% INSTRUCTIONS: Edit the file as appropriate and replace with your journal text. Do NOT edit
%							any section headers or titles, tabling commands, fonts, spacing, sizes, etc.

% -------  Do NOT edit this header
\documentclass[letterpaper,11pt]{article}
\usepackage[paperwidth=8.5in,left=1.0in,right=1.0in,top=1.0in,bottom=1.0in,paperheight=11.0in]{geometry}
\usepackage{graphicx}
\graphicspath{ {./} }
\usepackage{palatino}
\def\hrulefill{\leavevmode\leaders\hrule height 20pt\hfill\kern\z@}

% ------------- DO Edit these definitions ---------------------
\def\name{Mikail Khan}
\def\period{5}
\def\journalnum{20}
\def\daterange{3/2/19-3/6/19}
% ------------------ END ---------------------------------

% Do NOT edit this
\begin{document}
	\thispagestyle{empty}
	\begin{flushright}
		{\Large Journal Report \journalnum} \\
		\daterange\\
		\name \\
		Computer Systems Research Lab \\
		Period \period, White
		\end{flushright}
	\hrule height 1pt
% ------------------ END ---------------------------------%	
	
% ------------------- Begin Journal reporting HERE ---------------

% ------ SECTION DAILY LOG -------------------------------------
	\section*{Daily Log}

	\vspace{-0.5em}
		\subsection*{Monday March 2}
                Monday was a blue day but I adjusted collision stuff over the weekend. It seems like it works most of the time but sometimes things explode.
		\subsection*{Wednesday March 4}
                I looked into moving everything to the web. I already told you in person, but essentially what I found was that I'd have to write a graphics backend for the GUI library to work with WASM but it's likely that when the game engine officially supports WASM someone else will be pretty quick to write one. 
		\subsection*{Friday March 6}
                I looked more into WASM stuff and also fixed barrier positioning. I also looked into making a shader for the wavesim that combines groups of wavelets into a continous wave graphically. Right now there can only be around 1500 wavelets before it gets laggy, so I definitely think it's important to have a shader for it. As it turns out, WASM does support compute shaders, but the game engine doesn't without a bit of modding. I think I'll write a GPU accelerated gravity sim sometime but I don't think I'll add a GUI or anything.

	
% ------ SECTION TIMELINE -------------------------------------
	\newpage
	\section*{Timeline}
	\begin{tabular}{|p{1in}|p{2.5in}|p{2.5in}|}
		\hline 
	\textbf{Date} & \textbf{Goal} & \textbf{Met}\\ \hline
		\hline
		Today minus 2 weeks & Working wawve on wave sim & Almost\\
		\hline
		Today minus 1 weeks & Adjustable barriers & Not adjustable \\
		\hline
		Today  & Fix barriers and add creation UI & No, I kind of worked on other stuff\\
		\hline
		Today plus 1 week & Fix barriers and add creation UI &  \\
		\hline
		Today plus 2 weeks & Start working on wave generators &  \\
		\hline
	\end{tabular}


% ------ SECTION REFLECTION  -------------------------------------
	\section*{Reflection}
			In narrative style, talk about your work this week. Successes, failures, changes to timeline, goals. This should also
			include concrete data, e.g. snippets of code, screenshots, output, analysis, graphs, etc.

                        When I was making the weekly goals I didn't realize that this was a 2 day week. I started looking into WASM stuff because recently there's been a lot of commits in the game engine that are dedicated to it, so I thought I might finally be able to port. Wave on wave collision still looks kind of weird but it will be hard to tell if they're properly following wave laws until I add generators that can create waves at a set frequency.
\end{document}
